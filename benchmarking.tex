% The first command in your LaTeX source must be the \documentclass command.


\documentclass[sigconf]{acmart}
\settopmatter{printacmref=true}

\fancyhead{}
  % do not delete this code.

\usepackage{balance}
  % for creating a balanced last page (usually last page with references)

% defining the \BibTeX command - from Oren Patashnik's original BibTeX documentation.
\def\BibTeX{{\rm B\kern-.05em{\sc i\kern-.025em b}\kern-.08emT\kern-.1667em\lower.7ex\hbox{E}\kern-.125emX}}
    
% Rights management information. 
% This information is sent to you when you complete the rights form.
% These commands have SAMPLE values in them; it is your responsibility as an author to replace
% the commands and values with those provided to you when you complete the rights form.
%
% These commands are for a PROCEEDINGS abstract or paper.


% Submission ID. 
% Use this when submitting an article to a sponsored event. You'll receive a unique submission ID from the organizers
% of the event, and this ID should be used as the parameter to this command.
%\acmSubmissionID{123-A56-BU3}


% end of the preamble, start of the body of the document source.


\usepackage{algorithm}
\usepackage{algorithmic}
\usepackage{epsfig,url}
\usepackage{epsf}

%Control space between caption and figure/table
\captionsetup[figure]{font=small,skip=5pt}
%\captionsetup[table]{font=small,skip=0pt}


\copyrightyear{2021}
\acmYear{2021}
\setcopyright{acmcopyright}
\acmConference[ISPD '21]{Proceedings of the 2021 International Symposium on Physical Design}{March 29-April 1, 2021}{Virtual} \acmBooktitle{Proceedings of the 2021 International Symposium on Physical Design (ISPD '21), March 29-April 1, 2021, Virtual}
\acmPrice{15.00} \acmDOI{10.1145/3372780.3375563} \acmISBN{978-1-4503-7091-2/20/03}

\begin{document}


\title{Still Benchmarking After All These Years}
\iffalse
\author{Blind Review}
\else
\author{Jinwook Jung}
\email{jinwookjung@ibm.com}
\affiliation{
  \institution{IBM Watson Research Center}
}
\author{Patrick H. Madden}
\email{pmadden@binghamton.edu}
\affiliation{
  \institution{SUNY Binghamton Computer Science Department}
  \streetaddress{Box 6000}
  \city{Binghamton}
  \state{New York}
  \postcode{13902}
}
 
\fi
\begin{abstract}
Circuit benchmarks for VLSI physical design
have been growing in size and complexity, helping
the industry tackle new problems and find new
approaches.  In this paper, we take a look back
at how benchmarking efforts have shaped the
research community, consider trade-offs that
have been made, and speculate on what may come
next.


\end{abstract}

\begin{CCSXML}
\end{CCSXML}

\ccsdesc[500]{Hardware~Placement}


%
% Keywords. The author(s) should pick words that accurately describe the work being
% presented. Separate the keywords with commas.

\keywords{benchmarking; circuit placement; metrics}

\maketitle

\section{Introduction}

Integrated circuits have evolved at a breakneck pace for many years.  While
the early predictions by Moore\cite{Moore650114} might have seen far-fetched
at the time, Moore's Law has mostly held true.  Each year, we have more
transistors, they cost less, and all of this is on an exponential path.
To enable this growth, researchers in a range of fields have
seemingly endless innovations.  Physicists,
chemists, and process engineers have designed new devices. Lithographers
have found ways print mind-boggling levels of detail. Software developers
and algorithm designers have built new tools to handle the staggering
complexity.

Physical design automation researchers have played their part as well.
The placement and routing tools of the modern era are sophisticated,
letting circuit designers squeeze performance.
{\em Benchmarking} of tools and algorithms has been an essential
part of this growth.

In this paper, we look back at where the physical design community
was, some twenty years ago -- and consider what the research
community has done well, and what we might have done better, and
what we might want to do in the future.

Certainly, much has changed over the years.  Not so long ago, the
web was not ubiquitous -- getting access to benchmarks, even if they
were ``publicly available,'' required knowing who to ask, where to
look, and the navigation of anonymous FTP servers.  Disk space,
processor cycles, and memory, were all in short supply -- even with
access to benchmarks, it might not be possible to do anything with them.
In this era, nearly everyone has a smart phone with a super-computer
class processor, nearly infinite disk space, and vast amounts of
RAM on their workstations; it's easy to forget that many of the engineers
working today started with punch cards, paper terminals, and, if they
were lucky, floppy disks.


\section{Objectives of Benchmarking}
Why we do benchmarking.  This is all stuff that we ``know,'' but
may not have said explicitly.  And often, what we assume that we
all ``know'' is not in fact common knowledge.

\subsection{Better Solutions for Problems}
Industry can highlight critical problems, ones where new solutions
are needed.

\subsection{A Training Ground for Researchers}
Graduate students learn about physical design, hone their software
development skills.

\subsection{A Platform for Experimentation}
Fast turn-around invites higher-risk approaches, new ideas


\section{Getting Things Right}

Striking a balance between objectives.  Industry groups
would certainly like a production-ready tool, but this is
something that a small team of graduate students can't
build quickly.

Need to capture the essence of a problem, while keeping it
tractable.  Make the benchmark hard to game, so that solutions
actually resemble what we might want ``in practice.''

Trade-off on using library for parsing -- locks into a build
system, sometimes a language and set of tools.  Simple
file formats, by contrast, may lose the essential elements
that matter for an industrial design.

Evaluators, with painstaking detail, are important.

Complex tool flows for evaluation are trouble.

\section{The Bookshelf}

\cite{Caldwell000693}\cite{umichbookshelf}


\section{The ISPD Contests}

Beginning in 2005, ISPD has run annual contests to evaluate
different approaches to key physical design problems.  For circuit
placement, the contsts have occurred in 2005, 2006, 2011, 2014,
2015, with 2016 and 2017 focusing on placement for
field programmable gate arrays.

{\em talk about each year of the contest.  Metrics.  Citations
  for important publications based off the contest benchmarks.}

\subsection{Mixed Size Placement, 2005, 2006}

\subsection{Routability, 2011}

\subsection{Detail Routing Driven Placement, 2014, 2015}

\subsection{FPGA Placement 2016, 2017}

\subsection{Deep Learning Accelerator Placement, 2020}

\input natarajan.tex

\input jinwook.tex


\balance
\bibliographystyle{unsrt}
\bibliography{unify}


\end{document}

