% The first command in your LaTeX source must be the \documentclass command.


\documentclass[sigconf]{acmart}
\settopmatter{printacmref=true}

\fancyhead{}
  % do not delete this code.

\usepackage{balance}
  % for creating a balanced last page (usually last page with references)

% defining the \BibTeX command - from Oren Patashnik's original BibTeX documentation.
\def\BibTeX{{\rm B\kern-.05em{\sc i\kern-.025em b}\kern-.08emT\kern-.1667em\lower.7ex\hbox{E}\kern-.125emX}}
    
% Rights management information. 
% This information is sent to you when you complete the rights form.
% These commands have SAMPLE values in them; it is your responsibility as an author to replace
% the commands and values with those provided to you when you complete the rights form.
%
% These commands are for a PROCEEDINGS abstract or paper.


% Submission ID. 
% Use this when submitting an article to a sponsored event. You'll receive a unique submission ID from the organizers
% of the event, and this ID should be used as the parameter to this command.
%\acmSubmissionID{123-A56-BU3}


% end of the preamble, start of the body of the document source.


\usepackage{algorithm}
\usepackage{algorithmic}
\usepackage{epsfig,url}
\usepackage{epsf}

%Control space between caption and figure/table
\captionsetup[figure]{font=small,skip=5pt}
%\captionsetup[table]{font=small,skip=0pt}


\copyrightyear{2021}
\acmYear{2021}
\setcopyright{acmcopyright}
\acmConference[ISPD '21]{Proceedings of the 2021 International Symposium on Physical Design}{March 29-April 1, 2021}{Virtual} \acmBooktitle{Proceedings of the 2021 International Symposium on Physical Design (ISPD '21), March 29-April 1, 2021, Virtual}
\acmPrice{15.00} \acmDOI{10.1145/3372780.3375563} \acmISBN{978-1-4503-7091-2/20/03}

\begin{document}


\title{Still Benchmarking After All These Years}
\iffalse
\author{Blind Review}
\else
\author{Jinwook Jung}
\email{jinwookjung@ibm.com}
\affiliation{
  \institution{IBM Watson Research Center}
}
\author{Patrick H. Madden}
\email{pmadden@binghamton.edu}
\affiliation{
  \institution{SUNY Binghamton CSD}
  \streetaddress{Box 6000}
  \city{Binghamton}
  \state{New York}
  \postcode{13902}
}
\author{Natarajan Viswanathan}
\email{nviswan@cadence.com}
\affiliation{
  \institution{Cadence}
}


 
\fi
\begin{abstract}
Circuit benchmarks for VLSI physical design
have been growing in size and complexity, helping
the industry tackle new problems and find new
approaches.  In this paper, we take a look back
at how benchmarking efforts have shaped the
research community, consider trade-offs that
have been made, and speculate on what may come
next.


\end{abstract}

\begin{CCSXML}
\end{CCSXML}

\ccsdesc[500]{Hardware~Placement}


%
% Keywords. The author(s) should pick words that accurately describe the work being
% presented. Separate the keywords with commas.

\keywords{benchmarking; circuit placement; metrics}

\maketitle

\section{Introduction}

Integrated circuits have evolved at a breakneck pace for many years.  The
first MOSFET transistors were created at the end of the 1950's, with 
Moore\cite{Moore650114} making his bold prediction for exponential growth
in 1965.  While it might have seen far-fetched or wildly optimistic
at the time, Moore's Law has been a relentless juggernaut.
Each year, we have far more
transistors while costs plunge.
In 1960, only a handful of transistors could be integrated into a single
device, but by 1980, just twenty years later, the Motorola 68000 had
roughly sixty-eight thousand transistors.  Twenty years after that, the Intel Pentium 4
featured roughly forty-two million transistors.  Twenty years after that, in
the era in which this paper is being written, the Apple M1 processor has
in excess of sixteen billion transistors.

% https://en.wikipedia.org/wiki/Transistor_count

To enable this growth, researchers in a range of fields have
seemingly endless innovations.  Physicists,
chemists, and process engineers have designed new devices. Lithographers
have found ways print mind-boggling levels of detail. Software developers
and algorithm designers have built new tools to handle the staggering
complexity. 
Physical design automation researchers have played their part as well;
placement and routing tools of the modern era are sophisticated, effective,
and have tremendous scalability.

{\em Benchmarking}, to allow comparison of different approaches,
forms the foundation on which any progress is built.
In 1987, a panel discussion\cite{Preas87}
highlighted the need for good benchmarks for standard cell
design, and a few years later, the MCNC benchmarks\cite{Kozminski91}
were presented and quickly gained prominence.  Some of the benchmarks
grew out of discussions held at the Physical Design Workshop,
a precursor to ISPD.
While the
MCNC circuits were valuable, there were few updates -- and a
decade later, variations in how results were reported
stirred concern\cite{Madden010030}.

If the rapid progress demanded by Moore's Law was to continue, it
was essential that new benchmarks and better metrics be developed;
further, this could not be a one-off endeavor, but instead a
practice that would keep pace with industry trends.  The
GSRC supported GTX project
\cite{Caldwell000693} provided a platform to work from,
resulting in many industry and academic researchers
contributing their efforts to the GSRC Bookshelf\cite{umichbookshelf}.
New file formats, benchmarks, and tools, have become
a key part of the ISPD community.

In this paper, we look back at the impact of benchmarking
efforts on circuit placement, consider what the research
community has done well, and what
we might want to do in the future.


\section{Benchmarking in Physical Design}

Certainly, much has changed over the years.  Preas\cite{Preas87}
noted in 1987 that some circuits with over 10,000 cells were becoming
commonplace -- challenging to handle because representing them
``may require a substantial fraction of a megabyte for storage.''
Circuits of this size might seem absurdly small today -- but they
were in fact challenging.

It's easy to forget -- but not so long ago, the
world wide web was not ubiquitous. Getting access to benchmarks, even if they
were ``publicly available,'' required knowing who to ask, where to
look, and the navigation of anonymous FTP servers.  Disk space,
processor cycles, and memory, were all in short supply -- even with
access to benchmarks, it might not be possible to do anything with them.
In this era, nearly everyone has a smart phone with a blazingly fast
processor, nearly infinite disk space, and vast amounts of
RAM on their workstations.  Many of the engineers
working today started their careers with punch cards,
paper terminals, and, if they were lucky, floppy disks.

\subsection{Design Methodology}

A first step in building a platform for progress was the adoption
of a well defined design methodology -- namely standard cell design,
with lambda-based fabrication rules.  Mead and Conway were pioneers
and advocates\cite{Mead93}, shifting designers away from one-off
fully custom creations.  Standard cells and regular design rules allowed
for greater automation and interchangable design tools.

\subsection{File Formats and Methology}
% SIGDA newsletter! Franc Brglez from MCNC
% http://web.cs.ucla.edu/classes/layout/testing/Examples.iscas/ISCAS89/Benchmark.readme

{\bf Talk about production file formats versus things that
  are easy to parse.  Decisions made for GSRC Bookshelf.}

\cite{Caldwell000693}\cite{umichbookshelf}

\subsection{Benchmarking Objectives}

Why we do benchmarking.  This is all stuff that we ``know,'' but
may not have said explicitly.  And often, what we assume that we
all ``know'' is not in fact common knowledge.


\subsubsection{Better Solutions for Problems}

Industry can highlight critical problems, ones where new solutions
are needed.

Talk about gap between academic and industry designs, lack of
public access to leading edge stuff.  Industry needs competitive
advantage, holds information back.  Academic groups, even
with access, may be unable to publish.


\subsubsection{A Training Ground for Researchers}

All innovation is the product of men and women, working in both
academia and industry, around the world.  They get old, retire,
move into new areas.  Need a constant stream of new people to
carry on the work.

Well designed benchmarks give clear, tractable goals, and can
bring new people into the field.

Graduate students learn about physical design, hone their software
development skills.


\subsubsection{A Platform for Experimentation}

Fast turn-around invites higher-risk approaches, new ideas.

If it's only a few weeks worth of work, can try crazy
ideas, do things the ``wrong'' way, and stumble upon new
methods.

\subsubsection{Building the Community}

Level playing field, independent evaluation of results,
both public and private benchmarks avoid overtuning.

Opportunity for students in different research groups
to meet and interact, develop friendships and opportunities
for future collaboration.



\section{The ISPD Contests}

Beginning in 2005, ISPD has run annual contests to evaluate
different approaches to key physical design problems.  For circuit
placement, the contsts have occurred in 2005, 2006, 2011, 2014,
2015, with 2016 and 2017 focusing on placement for
field programmable gate arrays.

{\em talk about each year of the contest.  Metrics.  Citations
  for important publications based off the contest benchmarks.}

\subsection{Mixed Size Placement, 2005, 2006}

{\bf Gi-Joon}.  
Details here about ISPD 2005.  Cite relevant papers.
Benchmarks are adaptec, bigblue versions.  Legal2.pl script to
check legality.  Nine teams competing.
Fixed macros, from 200k to 2.1 million cells.

2006 adds target density, newblue designs.  Ten teams.  

\subsection{Routability, 2011}

\input natarajan.tex
Details, cite some papers.

\subsection{Detail Routing Driven Placement, 2014, 2015}

Details, cite some papers.
Detail Routing.
{\bf Ismail?}



\subsection{FPGA Placement 2016, 2017}

Details, cite some papers.
{\bf Stephen?}


\subsection{Deep Learning Accelerator Placement, 2020}

Include this?  Not really placement, but placement-ish?


\section{The Benchmarks Ahead}

Striking a balance between objectives.  Industry groups
would certainly like a production-ready tool, but this is
something that a small team of graduate students can't
build quickly.

Need to capture the essence of a problem, while keeping it
tractable.  Make the benchmark hard to game, so that solutions
actually resemble what we might want ``in practice.''

Trade-off on using library for parsing -- locks into a build
system, sometimes a language and set of tools.  Simple
file formats, by contrast, may lose the essential elements
that matter for an industrial design.

Evaluators, with painstaking detail, are important.

Complex tool flows for evaluation are trouble.







\subsection{The OpenROAD}

Talk about open source tool flow.  

\cite{Ajayi19}

% Section on RDF
\input jinwook.tex


\section{Concluding Comments}

Wrap up the paper with a feel-good paragraph or two.


\balance
\bibliographystyle{unsrt}
\bibliography{unify}


\end{document}

